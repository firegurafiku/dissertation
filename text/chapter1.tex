\chapter{Выравнивание волнового фронта в рупорах с линейным раскрывом}

\section{Искривление волнового фронта}

Если мы рассмотрим рассмотрим картину электромагнитного поля,
возникающую при излучение TEM-рупорной антенной сверхширокополосного
импульсного сигнала, то заметим, что пространственная конфигурация
импульса существенно изменяется по мере его прохождения от точки
запитки до выхода в свободное пространство.
Качественно этот процесс выглядит так:
%
\begin{itemize}
  \item
    до тех пор, пока импульс распространяется вдоль полосковой
    линии запитки, его форма практически не изменяется и близка
    к плоской (\FigRef{ch1:wavefrontDistortion:1});
  \item
    в точке соединения полосковой линии и лепестков антенны
    происходит частичное отражение сигнала, форма импульса
    начинает искажаться (\FigRef{ch1:wavefrontDistortion:2});
  \item
    по мере отдаления импульса от точки запитки искажение формы
    фронта волны возрастает, электромагнитное поле всё сильнее
    проникает за пределы пространства, лежащего между лепестками
    антенны (\FigRef{ch1:wavefrontDistortion:3}).
\end{itemize}

Картины полей, приведённый на~\FigRef{ch1:wavefrontDistortion},
получены при помощи электромагнитного моделирования простого
TEM-рупора, лепестки которого представляют собой плоские
равнобедренные трапеции (далee в работе мы будем называть подобные
антенны \FirstTerm{рупорами с линейным раскрывом}), 

\begin{figure}[p]
  \centering
  \begin{subfigure}[b]{0.3\linewidth}
    \ImagePlaceholder{7cm}
    \caption{\FUZZ{100~пс}}
    \label{ch1:wavefrontDistortion:1}
  \end{subfigure}
  %
  \begin{subfigure}[b]{0.3\linewidth}
    \ImagePlaceholder{7cm}
    \caption{\FUZZ{200~пс}}
    \label{ch1:wavefrontDistortion:2}
  \end{subfigure}
  %
  \begin{subfigure}[b]{0.3\linewidth}
    \ImagePlaceholder{7cm}
    \caption{\FUZZ{300~пс}}
    \label{ch1:wavefrontDistortion:3}
  \end{subfigure}

  % TODO: Explicit styling.
  \vspace{3mm}
  
  \begin{subfigure}[b]{0.3\linewidth}
    \ImagePlaceholder{7cm}
    \caption{\FUZZ{400~пс}}
  \end{subfigure}
  %
  \begin{subfigure}[b]{0.3\linewidth}
    \ImagePlaceholder{7cm}
    \caption{\FUZZ{500~пс}}
  \end{subfigure}
  %
  \begin{subfigure}[b]{0.3\linewidth}
    \ImagePlaceholder{7cm}
    \caption{\FUZZ{600~пс}}
  \end{subfigure}
  
  \caption{Искажение формы импульса по мере его распространения
    в рупорной антенне с линейным профилем раскрыва.}
  \label{ch1:wavefrontDistortion}
\end{figure}
